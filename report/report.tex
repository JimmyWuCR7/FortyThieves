\documentclass[12pt]{article}

\usepackage{graphicx}
\usepackage{paralist}
\usepackage{amsfonts}
\usepackage{amsmath}
\usepackage{hhline}
\usepackage{booktabs}
\usepackage{multirow}
\usepackage{multicol}
\usepackage{url}

\oddsidemargin -10mm
\evensidemargin -10mm
\textwidth 160mm
\textheight 200mm
\renewcommand\baselinestretch{1.0}

\pagestyle {plain}
\pagenumbering{arabic}

\newcounter{stepnum}

%% Comments

\usepackage{color}

\newif\ifcomments\commentstrue

\ifcomments
\newcommand{\authornote}[3]{\textcolor{#1}{[#3 ---#2]}}
\newcommand{\todo}[1]{\textcolor{red}{[TODO: #1]}}
\else
\newcommand{\authornote}[3]{}
\newcommand{\todo}[1]{}
\fi

\newcommand{\wss}[1]{\authornote{blue}{SS}{#1}}

\title{Assignment 4 Specification}
\author{Di Wu, 400117248, wud43}

\begin {document}

\maketitle
This Module Interface Specification (MIS) document contains modules, types and
methods for implementing the state of a game of Forty Thieves solitaire.

\wss{The parts that you need to fill in are marked by comments, like this one.
  In several of the modules local functions are specified.  You can use these
  local functions to complete the missing specifications.}

\wss{As you edit the tex source, please leave the \texttt{wss} comments in the
  file.  Put your answer \textbf{before} the comment.  This will make grading
  easier.}

\newpage

\section* {Board Module}

\subsection* {Generic Template Module}

Stack

\subsection* {Uses}

N/A

\subsection* {Syntax}

\subsubsection* {Exported Types}

Board = ?

\subsubsection* {Exported Constants}

None

\subsubsection* {Exported Access Programs}

\begin{tabular}{| l | l | l | p{5cm} |}
\hline
\textbf{Routine name} & \textbf{In} & \textbf{Out} & \textbf{Exceptions}\\
\hline
Board & Integer & Board & none\\
\hline
set & $\mathbb{N}$,$\mathbb{N}$ &  & none\\
\hline
PrintBoard & &  & none\\
\hline
NextStage & &  & none\\
\hline
CountNeigh & seq of (seq of $\mathbb{N}$) ,$\mathbb{N}$ ,$\mathbb{N}$ & $\mathbb{N}$ & none\\
\hline
toVec& & seq of (seq of $\mathbb{N}$) & none\\
\hline
\end{tabular}

\subsection* {Semantics}

\subsubsection* {State Variables}

None

\subsubsection* {State Invariant}

None

\subsubsection* {Assumptions \& Design Decisions}

\begin{itemize}
\item The Board constructor is called for each object instance before any
  other access routine is called for that object.  The constructor can only be
  called once.
\end{itemize}

\subsubsection* {Access Routine Semantics}

Stack($num$):
\begin{itemize}
\item transition: $S := BoardState[num][num]$

\item output: $\mathit{out} := \mathit{self}$
\item exception: none
\end{itemize}

\noindent set($a,b$):
\begin{itemize}
\item transition: $S[a][b] := 1$
\item exception: none
\end{itemize}

\noindent PrintBoard():
\begin{itemize}
\item exception: none
\end{itemize}

\noindent NextStage():
\begin{itemize}
\item transition: $(i,j \in [0...BoardState.size()] | (CountNeigh(BoardState,i,j) == 3 \and BoardState[i][j] == 0) \implies BoardState[i][j] = 1, (\not (CountNeigh(BoardState,i,j) == 3) \and BoardState[i][j] == 1) \implies BoardState[i][j] = 0)$

\item exception: none

\end{itemize}

\noindent CountNeigh(s,a,b):
\begin{itemize}
\item $out := (+1 | i \in [a-1 ... a+1] \and j \in [b-1 ... b+1] \and \not(i == a) \and \not (j == b) \and BoardState[i][j] == 1)$
\item exception: None
\end{itemize}

\noindent toVec():
\begin{itemize}
\item output: $\mathit{out} := BoardState$
\item exception: None

\end{itemize}

\newpage

\section*{Critique of Design}

\wss{Write a critique of the interface for the modules in this project.  Is there
anything missing?  Is there anything you would consider changing?  Why?}\\

Firstly, I believe that I created a function called toVec() in the BoardADT. My initial thoughts is that it will be convenient for achieving the other goals of other functions.
But it violates the information hiding policy. Secondly, in the Board constructor, I used a number as the input to create a num *num board. That is a little bit complex. 
Instead of doing it, I believe that I can directly use the content from the file-reading and build the BoardADT directly.

\end {document}